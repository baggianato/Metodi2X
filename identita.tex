\subsection*{Identità e relazioni}
\begin{gather*}
    ||z_1|-|z_2|| \le |z_1+z_2| \le |z_1|+|z_2| \\
    ||z_1|-|z_2|| \le |z_1-z_2| \le |z_1|+|z_2| \\
\end{gather*}

\begin{gather*}
    \cos z = \frac{e^{iz}+e^{-iz}}{2}; \
    \sin z  = \frac{e^{iz}-e^{-iz}}{2i};\\
    e^{\pm iz} = \cos z \pm i \sin z;\\
    \cosh{z} = \frac{e^{z}+e^{-z}}{2}; \
    \sinh{z}  = \frac{e^{z}-e^{-z}}{2}
\end{gather*}
\begin{definition}[Delta di Dirac]
    con centro nel punto $x_0$, indicata con $\delta_{x_0}$, oppure con $\delta(x-x_0)$, è una distribuzione così definita:
    \[
        \langle\delta_{x_0},\,\varphi\rangle\defeq\varphi(x_0),\ \forall\varphi\in\mathcal{F}.
    \]
    con notazione impropria, spesso si scrive
    \[
        \int_{-\infty}^{+\infty}\delta(x-x_0)\varphi(x)\,dx=\varphi(x_0).
    \]
\end{definition}
\begin{definition}[Parte principale $\mathcal{P}\frac{1}{x}$ (distribuzione)]
\begin{align*}
    \langle \pp{x}, \varphi \rangle &\equiv \mathcal{P}\int_{-\infty}^{+\infty}\frac{\varphi(x)}{x}dx=\\
    &=\lim_{\substack{R\to+\infty\\\epsilon\to0^+}}\Big\{\int_{-R}^{-\epsilon}+\int_{\epsilon}^{R}\Big\}\frac{\varphi(x)}{x}dx, \ \forall\varphi \in \mathcal{S}.
\end{align*}
\end{definition}
\begin{definition}[Funzione segno] Indicata anche con $\operatorname{sgn}(\w)$, si definisce
\[
    \varepsilon(\w) = 
    \begin{cases}
        +1\text{ per }\w>0;\\
        -1\text{ per }\w<0.
    \end{cases}
\]
\end{definition}
\begin{definition}[Distribuzioni $\delta_+$ e $\delta_-$]
\[
    \delta_\pm(x)\defeq\frac{1}{2}\delta(x)\pm\frac{1}{2\pi i}\pp{x}
\]
\end{definition}
\begin{gather*}
     \Theta(x) = \frac{1+\epsilon(x)}{2}\iff    \epsilon(x)=2\Theta(x)-1;\\
    \Theta(\w)+\Theta(-\w)=1\text{ e }\Theta(\w)-\Theta(-\w)=\epsilon(\w);\\
    \epsilon(-x) = -\epsilon(x);\\
    \lim_{\epsilon\to0^+}\frac{1}{x\pm\epsilon}=\pp{x}\mp i\pi\delta(x).
\end{gather*}

\begin{equation*}
    (\sqrt[n]{z})_k=\overline{\sqrt[n]{z}}e^{i(\frac{\arg (z)}{n}+\frac{2\pi k}{n})},\ k=0,\dots,n-1;\\
\end{equation*}

\begin{equation*}
    (\log (z))_k = \log|z|+i\arg(z) + i2\pi k, \ k\in\mathbf{Z};\\
\end{equation*}

\begin{align*}
    z^\alpha &= e^{\alpha\log|z|}e^{i\alpha\arg z}e^{i\alpha 2\pi k}=\\ 
    &= |z|^\alpha e^{i\alpha 2\pi k}, \quad k\in\mathbf{Z}, \ \alpha\in\mathbf{C}.
\end{align*}

\begin{equation*}
    \frac{1}{(z-a)(z+a)}=\frac{1}{2a}\Big[ \frac{1}{z-a}-\frac{1}{z+a}\Big]
\end{equation*}

\begin{equation*}
    \frac{1}{z(z+a)}=\frac{1}{a}\Big[ \frac{1}{z}-\frac{1}{z+a}\Big]
\end{equation*}

\subsubsection*{Serie note}
\[
    \sum_{n=0}^{\infty}\frac{1}{n!}\frac{1}{z^n} = e^{1/z}, \ |z|>0 \ (z\neq0).
\]
\[
    \sum_{n=0}^{\infty}\left(\frac{z}{a}\right)^n + \sum_{n=0}^{\infty}\left(\frac{b}{z}\right)^n = \frac{a}{a-z} + \frac{z}{z-b} \text{ per } |b|<|z|<|a|.
\]
\textbf{Serie geometrica}
\[
\sum_{n=0}^{\infty}x^n=
\begin{cases}
  \frac{1}{1-x},\text{ per }|x|<1;\\
  \text{divergente},\text{ per }x\ge1;\\
  \text{indeterminata},\text{ per }x\le-1.
\end{cases}
\]
equivalentemente
\[
    \sum_{n=0}^{\infty} z^n = \frac{1}{1-z},\ r = 1.
\]
La sua derivata
\[
    \sum_{n=1}^{\infty} nz^n = \frac{d}{dz}\left(\frac{1}{1-z}\right) = \frac{1}{(1-z)^2},\ r = 1.
\]
\[
    \sum_{n=0}^{\infty}(-1)^n x^n=  \frac{1}{1+x},\text{ per }|x|<1.
\]
Caso interessante:
\[
    \sum_{n=0}^{\infty} \frac{-1}{z^{n+1}}=\frac{1}{1-z}, \text{ per } |z|>1.
\]
\textbf{Serie esponenziale}
\[
    \sum_{n=0}^{\infty}\frac{z^n}{n!}=e^z, \ r=\infty.
\]

%%%%%%%%%%%%%%%%%%%%%%%%%%%%%%%%%%%%%%%%%%%%%%%%%%%%%%%%%%%%%%%%%%%%%%%%%%%%%%%%%%%%%%%%%%%%%%%%%%%%%%%
\begin{theorem}
    $f$ derivabile in senso complesso nel punto $z_0=x_0+iy_0 \Rightarrow f$ derivabile in senso reale, cioé $\exists\ f_x$ e $f_y$ nel punto $x_0$ e sono legate dalle condizioni di Cauchy Reimann $f_x+if_y=0$. 
\end{theorem}

\begin{theorem}
    $f(z)$ analitica in $D \subset \mathbf{C} \Leftrightarrow f(x,y)$ differenziabile in senso reale in $D$ con $f_x$ e $f_y$ continue e tali che $f_x+if_y=0$.
\end{theorem}
%%%%%%%%%%%%%%%%%%%%%%%%%%%%%%%%%%%%%%%%%%%%%%%%%%%%%%%%%%%%%%%%%%%%%%%%%%%%%%%%%%%%%%%%%%%%%%%%%%%%%%%
\subsubsection*{Condizioni di C.R.}
Coordinate cartesiane:
\[
   f_x +if_y=0\xRightarrow[]{f=u+iv}
  \begin{cases}
    u_x = v_y;y
    \\
    u_y = -v_x. 
  \end{cases}
\]
Coordinate polari:
\[
    f_{\rho} + \frac{i}{\rho} f_{\varphi}=0\xRightarrow[]{f=u+iv} 
    \begin{cases}
      u_{\rho}=\frac{1}{\rho}v_{\rho}; \\
      \frac{1}{\rho}u_\phi =-v_\rho.
    \end{cases}
\]

