\subsection*{Punti di diramazione}
\begin{definition}[Punti di diramazione]
Data una funzione $f(z)$, diremo che il punto $z_0$ si chiama \textbf{punto di diramazione} per la $f(z)$, se esiste un intorno del punto $z_0$ tale che ogni curva chiusa $\gamma$ contenuta in tale intorno e abbracciante $z_0$ gode della seguente proprietà: fissato un qualsiasi punto $z_1$ su $\gamma$ ed il corrispondente valore $f(z_1)$, facendo muovere $z$ lungo $\gamma$ a partire da $z_1$ e facendo variare con continuità il corrispondente valore della $f(z)$, tale valore \textit{non} assume il valore iniziale $f(z_1)$ quando il punto $z$ è tornato a coincidere con $z_1$ dopo aver percorso un intero giro u $\gamma$.
\end{definition}
\begin{definition}
Si chiama taglio una qualsiasi linea del piano complesso che congiunge i punti di diramazione, sulla quale si sceglie di posizionare la discontinuità della funzione.
\end{definition}
I punti di diramazione e i punti del taglio sono punti singolari \textit{non isolati} per la funzione.